\documentclass[a4paper]{report}
\usepackage{array}
\usepackage{hyperref}
\usepackage{makecell}
\usepackage{multicol}
\usepackage{multirow}
\usepackage{blindtext}
\usepackage{graphicx}
\usepackage{tabularx}
\usepackage{longtable}
\usepackage{booktabs}
\usepackage{ragged2e}
\setcounter{secnumdepth}{-1}
\begin{document}
\title{\TeX Live Accessibility Conformance Report\\
  (International Edition)\\
(Based upon VPAT\textsuperscript{\textregistered} version~2.4)}
\author{\TeX\ Users Group}
\date{May 2021}
\maketitle

\clearpage

\tableofcontents

\clearpage

\section{Name of Product/Version}
\label{sec:name}

\TeX Live 2021.


\section{Report Date}
\label{sec:date}

May 2021.


\section{Product Description}
\label{sec:description}

\TeX Live is the reference distribution of \TeX\ and related software
supported by \TeX\ Users Group, a membership-based not-for-profit
organization dedicated to support, promotion and advocacy of \TeX\
typesetting system created by Donald Knuth and maintained by the
international community of developers.  As of 2021 \TeX Live contains
more than 460~programs and more than 7300~software packages.

\section{Contact Information}
\label{sec:contact_info}

  \begin{description}
  \item[Website:]  \url{https://tug.org}
  \item[Address:] \leavevmode\\
    Robin Laakso, executive director,\\
    \TeX\ Users Group,\\
    PO Box 2311,\\
    Portland, OR~97208-2311,\\
    USA
  \item[Phone:] +1 503-223-9994
  \item[Fax:] +1 815-301-3568
  \item[Administrative email:] \href{mailto:office@tug.org}{office@tug.org}
  \item[Board of directors:] \href{mailto:board@tug.org}{board@tug.org}
  \item[President:] \href{mailto:president@tug.org}{president@tug.org}
  \end{description}

\newpage

\section{Notes}
\label{sec:note}

  
\TeX Live is a complex software suite, intended to the produce typeset
products in various formats (PDF, HTML, DVI, XML to name a few).  In
all case it takes an \emph{input code} written in a machine-readable
form and translates it to an \emph{output format,} suitable to
publication in the traditional (hard copy) or electronic form.

This document covers only the software used to convert the input code
to the output format.  The creation of the input code is outside the
scope of this document, since it is done by third party software.
\TeX Live software can accept as input text files created in any
authoring tool.  \TeX\ Users group is aware about fully accessible
authoring tools, and promotes them on its web pages and publications.

A separate issue is the accessibility of the \emph{output documents}
created with \TeX Live software.  The accessibility features of these
documents depend on the settings in the input code and the packages
used, and thus is not covered by this document.  \TeX\ Users Group
strives to make the creation of full accessible documents using its
tool as simple as possible, and to make the accessibility features the
default settings of our software.  This is an ongoing effort,
involving many developers.  \TeX\ Users Group welcomes any help in
this work.

The tools provided by \TeX Live are accompanied by technical
documentation, both included in \TeX Live itself, and separate: there
are many books, courses and other materials about \TeX.  Many of the
latter a fully accessible.  The main documentation of \TeX Live,
supported by \TeX Live team, is accessible as well.  Most of the
documentation for several thousand software packages in \TeX Live is
provided in PDF or text format.  By requiring machine readable textual
sources of all documentation provided by \TeX Live packages, we ensure
a certain level of accessibility for all documentation.  However, the
full audit of the documentation for all 7300+ packages created by many
thousands of volunteers is not feasible at this time.  \TeX\ Users
group supports the efforts of making all \TeX\ documentation fully
accessible.

\section{Evaluation methods used}
\label{sec:methods}

The testing was based on the knowledge of the general product
knowledge.

\section{Applicable standards and guidelines}
\label{sec:standards}

{\centering
\begin{tabularx}{\textwidth}{Xc}
  \toprule
  Standard/Guideline  & Included in the report\\
  \midrule
  \href{http://www.w3.org/TR/2008/REC-WCAG20-20081211}{Web Content
  Accessibility Guidelines 2.0} & Level A (Yes) \\
                      & Level AA (Yes)\\
                      & Level AAA (Yes)\\
  \href{https://www.w3.org/TR/WCAG21}{Web Content Accessibility
  Guidelines 2.1} & Level A (Yes) \\
  & Level AA (Yes)\\
  \href{https://www.access-board.gov/guidelines-and-standards/communications-and-it/about-the-ict-refresh/final-rule/text-of-the-standards-and-guidelines}{Revised Section 508 standards published January 18, 2017 and
  corrected January 22, 2018} & Yes\\
  \href{https://www.etsi.org/deliver/etsi_en/301500_301599/301549/03.01.01_60/en_301549v030101p.pdf}{EN
  301 549 Accessibility requirements suitable for public procurement
  of ICT products and services in Europe, - V3.1.1 (2019-11)} & Yes\\
  \bottomrule
\end{tabularx}
\par}

\section{Terms}
\label{sec:terms}

The terms used in the Conformance Level information are defined as follows:
\begin{description}
\item[Supports:] The functionality of the product has at least one
  method that meets the criterion without known defects or meets with
  equivalent facilitation.
\item[Partially Supports:] Some functionality of the product does not
  meet the criterion.
\item[Does Not Support:] The majority of product functionality does
  not meet the criterion.
\item[Not Applicable:] The criterion is not relevant to the product.
\item[Not Evaluated:] The product has not been evaluated against the
  criterion. This can be used only in WCAG 2.0 Level AAA.

\end{description}



\section{WCAG 2.x Report}
\label{sec:wcag}

Tables~1 and~2 also document conformance with:
\begin{itemize}
\item EN 301 549:  Chapter 9 - Web, Sections 10.1-10.4 of Chapter 10 - Non-Web documents, and Sections 11.1-11.4 and 11.8.2 of Chapter 11 - Non-Web Software (open and closed functionality), and Sections 12.1.2 and 12.2.4 of Chapter 12 – Documentation
\item Revised Section 508: Chapter 5 – 501.1 Scope, 504.2 Content
  Creation or Editing, and Chapter 6 – 602.3 Electronic Support
  Documentation.
\end{itemize}

\begin{description}
\item[Note:] When reporting on conformance with the WCAG 2.x Success
  Criteria, they are scoped for full pages, complete processes, and
  accessibility-supported ways of using technology as documented in
  the WCAG 2.0 Conformance Requirements.
\end{description}

\subsection{Introduction}
\label{sec:wcag-intro}

This section of the report describes the documentation for \TeX Live.

There are several ways to define ``the full documentation for \TeX\
and friends software''.  In one meaning of these words it comprises
all books on \TeX, \LaTeX, Con\TeX t and other systems included in
\TeX Live, all issues of the journals published by \TeX\ Users Group
and other users produced documentation.  Of course it would be
impossible to estimate the accessibility of all this documentation,
comprising hundreds thousands of pages written over three decades.

Another way is to count only the documentation for the packages
included in \TeX Live.  This includes documentation for 7300+
packages, which makes the full audit rather impractical.

In this document we discuss only the main manual for the system,
available with the installation and at
\url{https://tug.org/texlive/doc/texlive-en/texlive-en.html}.  We
note, however, that documentation for all \TeX Live packages is
available as source text files, and thus is highly accessible.  


In the tables below the success criteria apply to \emph{Web} and
\emph{Electronic Docs} paragraphs of VPAT.


\section{Table~1:  Success Criteria, Level A}
\begin{longtable}{p{0.5\textwidth}<{\RaggedRight}p{0.2\textwidth}<{\RaggedRight}p{0.2\textwidth}<{\RaggedRight}}
  \toprule
  Criteria & Conformance Level & Remarks and Explanations \\
  \midrule
  \endhead
  \bottomrule
  \endfoot
	\href{https://www.w3.org/TR/WCAG20/#text-equiv-all}{1.1.1
  Non-text Content (Level A)} & Partially supports & The figures
                                                     describing GUI do
                                                     not have alt
                                                     text\\
        \href{https://www.w3.org/TR/WCAG20/#media-equiv-av-only-alt}{1.2.1
  Audio-only and Video only (Prerecorded) (Level A)} & Not Applicable\\
  \href{https://www.w3.org/TR/WCAG20/#media-equiv-captions}{1.2.2
  Captions Prerecorded} (Level A) & Not Applicable\\
        \href{http://www.w3.org/TR/WCAG20/#media-equiv-audio-desc}{1.2.3
                                    Audio Description or Media
                                    Alternative (Prerecorded)}
                               & Not Applicable\\
        \href{http://www.w3.org/TR/WCAG20/#content-structure-separation-programmatic}{1.3.1
  Info and Relationships} (Level A) & Supports \\
    \href{http://www.w3.org/TR/WCAG20/#content-structure-separation-programmatic}{1.3.2
  Meaningful Sequence} (Level A) & Supports \\
  \href{http://www.w3.org/TR/WCAG20/#content-structure-separation-understanding}{1.3.3
  Sensory Characteristics} (Level A) & Not Applicable\\
  \href{http://www.w3.org/TR/WCAG20/#visual-audio-contrast-without-color}{1.4.1
  Use of Color} (Level A) & Supports \\
  \href{http://www.w3.org/TR/WCAG20/#visual-audio-contrast-dis-audio}{1.4.2
  Audio Control} (Level A) & Not Applicable\\
  \href{http://www.w3.org/TR/WCAG20/#keyboard-operation-keyboard-operable}{2.1.1
  Keyboard} (Level A) & Supports \\
  \href{http://www.w3.org/TR/WCAG20/#keyboard-operation-trapping}{2.1.2
  No Keyboard Trap} (Level A) & Supports \\
  \href{https://www.w3.org/TR/WCAG21/#character-key-shortcuts}{2.1.4
  Character Key Shortcuts} (Level A 2.1 only) & Supports \\
        \href{http://www.w3.org/TR/WCAG20/#time-limits-required-behaviors}{2.2.1
  Timing Adjustable} & Supports\\
  \href{http://www.w3.org/TR/WCAG20/#time-limits-pause}{2.2.2 Pause, Stop,
  Hide} (Level A) & Supports\\
  \href{http://www.w3.org/TR/WCAG20/#seizure-does-not-violate}{2.3.1
  Three Flashes or Below Threshold} (Level A) & Supports\\
  \href{http://www.w3.org/TR/WCAG20/#navigation-mechanisms-skip}{2.4.1
  Bypass Blocks} & Not Applicable\\
  \href{http://www.w3.org/TR/WCAG20/#navigation-mechanisms-title}{2.4.2 Page Titled} (Level A) & Supports\\
  \href{http://www.w3.org/TR/WCAG20/#navigation-mechanisms-focus-order}{2.4.3 Focus Order} (Level A) & Supports\\
  \href{http://www.w3.org/TR/WCAG20/#navigation-mechanisms-refs}{2.4.4 Link Purpose (In Context)} (Level A) & Supports\\
  \href{https://www.w3.org/TR/WCAG21/#pointer-gestures}{2.5.1 Pointer Gestures} (Level A 2.1 only) & Supports\\
  \href{https://www.w3.org/TR/WCAG21/#pointer-cancellation}{2.5.2
  Pointer Cancellation} (Level A 2.1 only) &  Supports\\
  \href{https://www.w3.org/TR/WCAG21/#label-in-name}{2.5.3 Label in Name} (Level A 2.1 only) & Supports\\
  \href{https://www.w3.org/TR/WCAG21/#motion-actuation}{2.5.4 Motion Actuation} (Level A 2.1 only) & Supports\\
  \href{http://www.w3.org/TR/WCAG20/#meaning-doc-lang-id}{3.1.1 Language of Page} & Supports\\
  \href{http://www.w3.org/TR/WCAG20/#consistent-behavior-receive-focus}{3.2.1 On Focus} (Level A) & Supports\\
  \href{http://www.w3.org/TR/WCAG20/#consistent-behavior-unpredictable-change}{3.2.2 On Input} (Level A) & Supports\\
  \href{http://www.w3.org/TR/WCAG20/#minimize-error-identified}{3.3.1 Error Identification} (Level A) & Supports\\
        \href{http://www.w3.org/TR/WCAG20/#minimize-error-cues}{3.3.2 Labels or Instructions} (Level A) & Supports\\
  \href{http://www.w3.org/TR/WCAG20/#ensure-compat-parses}{4.1.1 Parsing} (Level A) & Supports\\
  \href{http://www.w3.org/TR/WCAG20/#ensure-compat-rsv}{4.1.2 Name, Role, Value} & Supports\\
\end{longtable}

\section{Revised Section~508 Report}
\label{sec:sec508}

\subsection{Chapter 3: Functional Performance Criteria (FPC)}
\label{sec:508-3}


\end{document}

%%% Local Variables:
%%% mode: latex
%%% TeX-master: t
%%% End:
