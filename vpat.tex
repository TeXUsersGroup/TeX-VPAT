\documentclass{report}

\usepackage{hyperref}
\setcounter{secnumdepth}{-1}
\begin{document}
\title{\TeX Live Accessibility Conformance Report\\
  (International Edition)\\
(Based upon VPAT\textsuperscript{\textregistered} version~2.4)}
\author{\TeX\ Users Group}
\date{April 2021}
\maketitle

\clearpage

\tableofcontents

\clearpage

\section{Name of Product/Version}
\label{sec:name}

\TeX Live 2021.


\section{Report Date}
\label{sec:date}

April 2021.


\section{Product Description}
\label{sec:description}

\TeX Live is the reference distribution of \TeX\ and related software
supported by \TeX\ Users Group, a membership-based not-for-profit
organization dedicated to support, promotion and advocacy of \TeX\
typesetting system created by Donald Knuth and maintained by the
international community of developers.  As of 2021 \TeX Live contains
more than 460~programs and more than 7300~software packages.

\section{Contact Information}
\label{sec:contact_info}

  \begin{description}
  \item[Website:]  \url{https://tug.org}
  \item[Address:] \leavevmode\\
    Robin Laakso, executive director,\\
    \TeX\ Users Group,\\
    PO Box 2311,\\
    Portland, OR~97208-2311,\\
    USA
  \item[Phone:] +1 503-223-9994
  \item[Fax:] +1 815-301-3568
  \item[Administrative email:] \href{mailto:office@tug.org}{office@tug.org}
  \item[Board of directors:] \href{mailto:board@tug.org}{board@tug.org}
  \item[President:] \href{mailto:president@tug.org}{president@tug.org}
  \end{description}


\section{Notes}
\label{sec:note}

  
\TeX Live is a complex software suite, intended to the produce typeset
products in various formats (PDF, HTML, DVI, XML to name a few).  In
all case it takes an \emph{input code} written in a machine-readable
form and translates it to an \emph{output format,} suitable to
publication in the traditional (hard copy) or electronic form.

This document covers only the software used to convert the input code
to the output format.  The creation of the input code is outside the
scope of this document, since it is done by third party software.
\TeX Live software can accept as input text files created in any
authoring tool.  \TeX\ Users group is aware about fully accessible
authoring tools, and promotes them on its web pages and publications.

A separate issue is the accessibility of the \emph{output documents}
created with \TeX Live software.  The accessibility features of these
documents depend on the settings in the input code and the packages
used, and thus is not covered by this document.  \TeX\ Users Group
strives to make the creation of full accessible documents using its
tool as simple as possible, and to make the accessibility features the
default settings of our software.  This is an ongoing effort,
involving many developers.  \TeX\ Users Group welcomes any help in
this work.

The tools provided by \TeX Live are accompanied by technical
documentation, both included in \TeX Live itself, and separate: there
are many books, courses and other materials about \TeX.  Many of the
latter a fully accessible.  The main documentation of \TeX Live,
supported by \TeX Live team, is accessible as well.  Most of the
documentation for several thousand software packages in \TeX Live is
provided in PDF or text format.  By requiring machine readable textual
sources of all documentation provided by \TeX Live packages, we ensure
a certain level of accessibility for all documentation.  However, the
full audit of the documentation for all 7300+ packages created by many
thousands of volunteers is not feasible at this time.  \TeX\ Users
group supports the efforts of making all \TeX\ documentation fully
accessible.  



\end{document}

%%% Local Variables:
%%% mode: latex
%%% TeX-master: t
%%% End:
