\documentclass{ltugboat}
\usepackage{microtype, siunitx}
\usepackage[breaklinks,hidelinks]{hyperref}
\title{Creating \TeX Live \acro{VPAT}\textsuperscript{\textregistered} statement}
\author{Boris Veytsman}
\netaddress{borisv (at) lk (dot) net}
\personalURL{http://borisv.lk.net}
\address{George Mason University (US),  \TeX\ Users Group}
\author{Keiran Harcombe}
\netaddress{kjh (at) harcombe (dot) net}
\personalURL{https://harcombe.net/~keiran/contact}
\address{Open University (UK)}
\begin{document}
\maketitle

\begin{abstract}
  Governments around the world are enforcing accessibility standards.
  Vendors of software used by government agencies are required to file
  formal statements of accessibility for their products.  This
  presents a special challenge of open source products, if they are
  not sponsored by a corporation.

  In this paper we discuss our experience in creation of such a formal
  statement for \TeX Live.  While command line tools are usually more
  accessible than \acro{GUI} interfaces, the work turned out to be
  more difficult than we thought in the beginning.
\end{abstract}

\section{Introduction}
\label{sec:introduction}

If we want to prove that the current scientific and technical progress
is accompanied by a moral progress, then the changing consensus about
the people with disabilities may provide a solid argument.  We now
more and more think that the full participation of disadvantaged
people in the life of society is not a generous gift from the society
to them, but rather their right.  Governments now often act to protect
this right.  In particular, according to the regulations, software
vendors must make reasonable efforts to make their products
accessible, and to formally report these efforts.  In many countries
including \acro{US} and \acro{EU} these reports are becoming a
condition for the software to be used by government agencies and
contractors.  

This presents a challenge to the free software.  Commercial vendors
have resources to hire lawyers and experts to create formal compliance
reports.  Free software like \TeX\ is often created by volunteers
across the world.  Volunteers are motivated to do ``interesting''
things rather than wade through pages of regulations and templates.
On the other hand, the absence of the reports can undermine the usage
of free software by the public sector and anybody dealing with the
government agencies, which in the modern society means everybody.  The
organizations like \tug\ might be useful in this situation, since
``uninteresting'' work necessary for the free software to thrive is
clearly in their remit.

In this paper we discuss a pilot project in this vein:  \TeX Live
\acro{VPAT}\textsuperscript{\textregistered} statement.

\section{Rules, regulations, templates}
\label{sec:rules}

A template for compliance report was developed by the Information
Technology Industry Council
(\acro{ITI})\footnote{\url{https://www.itic.org/}}.  It is available as
Voluntary Product Accessibility Template
(\acro{VPAT}\textsuperscript{\textregistered})\footnote{\url{https://www.itic.org/policy/accessibility/vpat}}
and is free to use.  The template is based on several sources:
\begin{enumerate}
\item Web Content Accessibility Guidelines developed by W3 Consortium
  (\acro{WCAG})\footnote{\url{http://www.w3.org/TR/2008/REC-WCAG20-20081211}
    and \url{https://www.w3.org/TR/WCAG21}}.
\item Revised Section 508 Standards by \acro{US}
  Government\footnote{\url{https://www.access-board.gov/guidelines-and-standards/communications-and-it/about-the-ict-refresh/final-rule/text-of-the-standards-and-guidelines}}.
\item EN~301~549 Accessibility requirements suitable for public
  procurement of \acro{ICT} products and services in
  Europe\footnote{\url{https://www.etsi.org/deliver/etsi_en/301500_301599/301549/03.01.01_60/en_301549v030101p.pdf}}.
\end{enumerate}
Note that \acro{WCAG} is not, strictly saying, a government standard.
However, both \acro{US} and European standards require compliance with
\acro{WCAG}, or, in lawyers' lingo, incorporate it.  

\acro{ITI} template has several variants, or ``editions''.  Since
\TeX\ is used throughout the world, we chose the largest one, the
International edition, that reports the compliance with both \acro{US}
and \acro{EU} standards.

The standards consider several following categories of software.
\emph{Web documents} are documents published on \acro{WWW}, while
\emph{electronic documents} are documents published in any other
manner.  Even if a product is not a document by itself, its
documentation must be accessible and thus comply with \acro{WCAG}
standards.  \emph{Software} is a generic term for software. The term
\emph{Closed system} does not mean what we in the Free software mean
by these words: the standards define closed systems as those which do
not allow an easy interaction with assistive technologies.  Lastly,
\emph{Authoring tool} is a software for creation of documents.  The
compliance report may discuss compliance with any or all of these
categories, with one exception: documentation is usually a part of any
category.

\section{Our decisions and the lessons learned}
\label{sec:decisions}

Initially we thought that our task was easy.  \TeX\ and friends are
command line tools.  We know that command line tools are very
accessible: people with low hearing, low vision or blind can use
command line tools with the existing assistive technology.  However,
when we got better acquainted with the standards and the requirements,
we understood this was not as easy as seemed.

First, the compliance report cannot be generic.  We cannot say that
any implementation of \TeX\ and its ecosystem is accessible.
Therefore we decided to choose at first a concrete implementation,
which could be installed and operated with assistive technologies.  We
chose \TeX Live 2021: our flagship implementation of \TeX\ and
friends.  We plan to add other implementations to the list in the
future.

Second, we must report whether the documentation for \TeX\ is
compliant.  This immediately causes a question, what is \TeX\
documentation?  At the point we counted it, \TeX Live 2021 had
\num{36769}~files in the \texttt{doc} tree in various formats: mostly
text, PostScript, \acro{PDF} and \acro{HTML}.  Obviously many of them
are not accessible.  Fortunately it is not needed to read them
\emph{all} to successfully use \TeX Live.  To tell the truth, on
practice the situation is ameliorated by long standing policy of \TeX
Live to require source code for all its documentation:  text source
code can be read by voice software.  

We decided to discuss only one piece of documentation: \TeX Live
manual in \acro{HTML} format at
\url{https://www.tug.org/texlive/doc/texlive-en}.
\acro{WCAG} guidelines define three levels of compliance, denoted as
\acro{A}, \acro{AA}, and \acro{AAA}.  Only the first two are required
by the \acro{US} and \acro{EU} standards.  The third one contains
rather difficult requirements such as automatic explanation of
abbreviations and special terminology.  However, we found problems
even on the first two levels.  Namely, the guidelines require
alternative text for all images.  \TeX Live manual has images for
\acro{GUI} installation mode.  One can argue that these images are
not especially relevant for low vision users, who are probably going
to use text mode installation anyway.  However, this is a weak
argument, and we need to make the manual fully accessible.

Further, the category \emph{closed systems} is evidently not
applicable for \TeX Live: it is open not only as open source software,
but also as software which allows the user to integrate it with the
assistive technologies.

We hit a snag, however, when looking into the requirements for
\emph{authoring tools}.  Simply saying, these requirements say that
the software is not just accessible by itself, but also the documents
it creates are accessible.  Now \tug\ and the \TeX\ community has put
considerable efforts in creating the tools for accessible output, like
tagged \acro{PDF}.  However, at present the creation of accessible
documents with \TeX, while possible, still requires certain efforts.
Thus we made a decision not yet to make a statement about \TeX Live as
an authoring tool.  Simply saying, we state that, for example, a
blind person can efficiently learn and use \TeX\ to create a
document.  However we do \emph{not} state that this document will be
accessible even to its author.  This is a sad situation, but hopefully
it will be ameliorated soon.

The resulting draft statement is available at
\url{https://github.com/TeXUsersGroup/TeX-VPAT}.

\section{Next steps}
\label{sec:next_steps}

Our work showed that there is much to do with \TeX\ compliance
documentation.

First, we must make \TeX Live manual fully compliant by adding alt
text to all images.

Second, we need to make the statement itself fully accessible, either
in \acro{PDF} or \acro{HTML} format, or both.

Third, we need to add separate statements for other distributions:
Mac\TeX, Mik\TeX, pro\TeX t\ldots\  We would like to invite volunteers
best acquainted with these distributions to help with them:  we hope it
might be easy enough to follow our example.

Fourth, it is important to add the statement about \TeX\ as an
authoring tool.  Perhaps we may put the statements about Con\TeX t,
which currently has good capabilities for creation of tagged
\acro{PDF}, or the usage of \textsl{tex4ht} for creation of accessible
\acro{HTML} files.  This requires further research and decisions.

The task of making all \TeX Live documentation accessible seems to be
very difficult, but we might start with the guidelines for package
authors.

Last but not least, we plan to release the statement as a \acro{CTAN}
package.

Making \TeX\ accessible is an ongoing process.  We are glad to be a
part of it.

\makesignature
\end{document}

%%% Local Variables:
%%% mode: latex
%%% TeX-master: t
%%% End:
