\documentclass{beamer}
\usepackage{beamertheme-light}
\usepackage{siunitx}
\setlength{\parskip}{\baselineskip}

% Increase footline height to accommodate superscripts
\setbeamertemplate{footline}
{
  \leavevmode%
  \hbox{%
  \begin{beamercolorbox}[wd =.45\paperwidth, ht = 2.75ex, dp = 1ex, center]{author in head/foot}%
    \usebeamerfont{author in head/foot}\colorbox{structure.bg}{\makebox[0.45\paperwidth]{\vphantom{T\textsuperscript{\textregistered}}\insertshortauthor}}
  \end{beamercolorbox}%
  \begin{beamercolorbox}[wd =.45\paperwidth, ht = 2.75ex, dp = 1ex, center]{title in head/foot}%
    \usebeamerfont{title in head/foot}\colorbox{structure.bg}{\makebox[0.45\paperwidth]{\insertshorttitle}}
  \end{beamercolorbox}%
  \begin{beamercolorbox}[wd=.10\paperwidth, ht = 2.75ex, dp = 1ex, right]{date in head/foot}%
    \usebeamerfont{date in head/foot}\colorbox{structure.bg}{\makebox[0.10\paperwidth]{
    \vphantom{T\textsuperscript{\textregistered}}\insertframenumber{} / \inserttotalframenumber\hspace*{2ex}}}
  \end{beamercolorbox}}%
  \vskip0pt%
}


\begin{document}
\author{Boris Veytsman\thanks{George Mason University (US),  \TeX\ Users Group }
  \and
Keiran Harcombe\thanks{Open University (UK)}}

\title{Creating \TeX Live VPAT\textsuperscript{\textregistered} statement}
\date{TUG~2021}


\begin{frame}
  \maketitle
\end{frame}

\begin{frame}
  \frametitle{What is VPAT\textsuperscript{\textregistered} and why do
    we need it?}

  
  European and US laws require accessibility features in hardware and
  software.  Government agencies \& contractors need a formal
  statement about conformance.  Otherwise they may not be able to use
  a system, \emph{including \TeX}.  \emph{[Yes, we know that many
    people with disabilities use \TeX, and we are proud of it.  No, it is
    not enough for the governments.]}

  \href{https://www.itic.org/}{The Information Technology Industry
    Council (ITI)} developed
  \href{https://www.itic.org/policy/accessibility/vpat}{Voluntary
    Product Accessibility Template
    (VPAT\textsuperscript{\textregistered})} for this statement.  

  \TeX\ itself: a huge ecosystem with hundreds of programs.  We can
  \emph{not} offer an accessibility statement for everything.
  Solution: offer separate statements for \TeX\ distributions.

  \emph{Currently this means \TeX Live.}

  
\end{frame}


\begin{frame}
  \frametitle{Standards and sources}

  \begin{description}
  \item[WCAG:] \href{http://www.w3.org/TR/2008/REC-WCAG20-20081211}{Web
      Content Accessibility Guidelines 2.0} and
    \href{https://www.w3.org/TR/WCAG21}{Web Content Accessibility
      Guidelines 2.1}.  Developed for accessibility of Web pages.
    Used to access accessibility of electronic documentation
    \emph{including software documentation.}
    \item[Sec 508:]
      \href{https://www.access-board.gov/guidelines-and-standards/communications-and-it/about-the-ict-refresh/final-rule/text-of-the-standards-and-guidelines}{Revised
        Section 508 Standards}.  US Government standard.
      Incorporates WCAG.
\item[EN~301~549:] \href{https://www.etsi.org/deliver/etsi_en/301500_301599/301549/03.01.01_60/en_301549v030101p.pdf}{EN
  301 549 Accessibility requirements suitable for public procurement
  of ICT products and services in Europe.}  European
standard. Incorporates WCAG.
\end{description}

There are several versions of VPAT\textsuperscript{\textregistered}.
Since our distribution is international, we chose \emph{international
  edition}.  

\end{frame}

\begin{frame}
  \frametitle{What is our document is and is not}

  \begin{enumerate}
  \item We do \emph{not} state that the \emph{end product} is
    accessible or even that one \emph{can} create an accessible
    product with \TeX\ (but see later!)
  \item We \emph{do} state that \TeX\ itself is accessible:  a person
    with certain disabilities can
    \begin{itemize}
    \item Learn how to use \TeX Live, and
    \item Efficiently use \TeX Live.
    \end{itemize}

  \end{enumerate}
  
\end{frame}

\begin{frame}
  \frametitle{Initial idea (and why it was wrong)}

  \begin{description}
  \item[Initial thought:] Our tools are command line. Command line is
    accessible  $\Rightarrow$ our tools are accessible.  Case
    closed.  
  \item[Problems] (found when we started the effort):
    \begin{enumerate}
    \item Accessible products must have accessible documentation.  Is
      our documentation accessible?
    \item Accessible products must satisfy certain criteria.  We must
      explicitly list them and state the conformance.
    \end{enumerate}

  \end{description}
  
\end{frame}

\begin{frame}
  \frametitle{\TeX Live documentation and WCAG standards}
  \emph{Accessible products must have accessible documentation}.  But
  current \TeX\ Live has \num{36769} documents.  Are they all
  accessible? Can we verify this?

  The current decision: certify only the main document,
  \emph{\href{https://www.tug.org/texlive/doc/texlive-en/texlive-en.html}{The
      TeX Live Guide—2021}}.

  Results:
  \begin{enumerate}
  \item Mostly AA~level compliance.
  \item We need to provide alt text for all images---even if they are
    not used in text only regime!
  \item Providing AAA~level compliance is difficult, and not needed by
    government standards.
  \end{enumerate}

\end{frame}

\begin{frame}
  \frametitle{\TeX Live \& government standards}

  US Government and European Union Standards are different for
  different types of programs:
  \begin{description}
  \item[Web:]  For us it is Web documentations.
  \item[Electronic Docs:] For us it is our documentation.
  \item[Software:] Generic programs.
  \item[Closed systems:] Systems that cannot be connected to
    accessibility systems (screen readers etc)
  \item[Authoring tools:] Tools used to create documents.  
  \end{description}

  We state compliance for Web and Non-web documentation, and software
  with one exception: \emph{Usage with limited cognition, language or
    learning.}  
  
\end{frame}

\begin{frame}
  \frametitle{Results and next steps}
  The result:  \url{https://github.com/TeXUsersGroup/TeX-VPAT}.

  Next steps (from easy to difficult):
  \begin{enumerate}
  \item We need to make \TeX Live documentation fully compliant
    $\Rightarrow$ add alt text to images.
  \item We need to make VPAT statement itself fully accessible.
  \item We may want to add statements for other distributions:
    Mac\TeX, Mik\TeX, pro\TeX t\ldots
  \item We need to add certification of \TeX\ as authoring tool.  This
    means tagging PDF.
  \item We might start making package documentation accessible.  
  \end{enumerate}

\end{frame}

\end{document}

%%% Local Variables:
%%% mode: latex
%%% TeX-master: t
%%% End:
